
Agent-based models provide a method of simulating investor behaviour in an electricity market. Using this technique, we were able to observe the emergent investor behaviour. 

We observed that an increase in carbon tax had a significant impact on investment. These findings enable policy makers to better understand the impact that their decisions may have. For a high uptake of renewable energy technology, rapid results can be seen after 10 years with a carbon tax of \textsterling70.

Agent-based models open up the possibility of testing differing investor behaviours through techniques such as reinforcement learning. This can be extended to incorporate collusion which can have an impact in liberalized electricity markets \cite{Benjamin2016}.

We propose the integration of a higher temporal and spatial resolution to model the utility of batteries, distributed generation and scarceness in renewable resources such as wind and solar at certain times of the day.


\FloatBarrier