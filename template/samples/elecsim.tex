
ElecSIM has been designed for ease of use to enable non-experts to rapidly test policy ideas. The core to this has been the design of a scenario file. This single file allows exogenous variables such as fuel cost, carbon taxes, power plants to be initialized, power plant costs, electricity demand and availability factors to be set. This allows for the initialisation of different countries and different scenarios to be tested.

ElecSIM is made up of 2 different agent types, GenCos (Generation Companies) and a Demand agent. GenCos can be initialised as to the wishes of the user. Either according to real life, or a toy example. Each of the GenCos are initialised with their respective power plants. GenCos are also given a randomized discount rate, which can be set by the user, around a mean of 10\% for nuclear power plants \cite{Paper2012} and 5.9\% for all other generators \cite{KPMG2017}.

ElecSIM's power generation costs is initialised using the UK government Department for Business and Energy power plant generation report \cite{Department2016}. This contains information such as capital costs and operation and maintenance costs, including details such as insurance and connection costs. Where there are power plants of a size that is not included in the  costs the parameters are linearly interpolated. Where the capacity of a power plant is larger or smaller than the data points in the report, the parameters are extrapolated by using the last known data point.

For historical power plants, we used historical levels of levelised cost of energy (LCOE). Each parameter was scaled linearly from the modern LCOE, to attain the relevant historical LCOE. This was achieved by using linear optimisation, and therefore can be changed based on an individual user's country and dataset by customising the contraints. As well as historical LCOE, historical plant efficiency was taken into account for Gas and Coal power plants.

When initialised, the variable operation and maintenance costs are selected from a uniform distribution, which can be set by the user. This enables variance in costs between individual power plants for processes such as preventative and corrective maintenance, labour costs and skill, health and safety and chance.  

As per \cite{Chappin2017}, we split the electricity demand for one year and split this into 19 load segments. This enabled us to model the varying demand of electricity throughout a year, and at the same time reduce computational complexity. To model the Intermittency of wind and solar power we allow them to contribute only a certain percentage of total capacity for each load segment based on empirical wind and solar capacity factors, relating demand to average capacity \cite{Pfenninger2016, Staffell2016, Chappin2017}.

Whilst fuel price is controlled by the user, there is inherent volatility in fuel price in a single year. To take into account this variability, an ARIMA model was fit to historical gas and coal price data \cite{coalprices,gasprices}. The standard deviation of the residuals was taken and then used to model the price of fuel that a generation company will buy fuel at in a given year. This takes into account differences in hedging strategies and chance between generation companies.




\begin{itemize}
	\item Model can be modified through a single python scenario file which includes exogenous variables such as number of generation companies, power plants, power plant costs, tax and fuel prices, and demand.
	\item Architectural framework:
	\begin{itemize}
		\item Agents are generation companies.
		\item Generation companies initialized from government data. And randomized discount rate around a mean of 10\% for nuclear power plants and 5.9\% for other types of generators.
		\item Costs of power plants taken from empirical data. 
		\item Historical LCOE costs taken from data, with individual costs such as fixed operation and maintenance, construction and pre-development costs scaled linearly to match LCOE value. (This can be changed by user by specifying linear optimisation constraints).
		\item Historical Gas turbine and Coal plant efficiency taken from epa data.
		\item Variable operation and maintenance costs are stochastic to take into account differences in design types, preventative and corrective maintenance, labour costs and skill, asset and site management, health and safety and chance.
		\item Electricity demand taken from historical data and split up into 19 load segments.
		\item CO2 prices, fuel Prices, demand growth are exogenous
		\item Fuel is bought by power producers each year at different prices, related to the standard deviation from historical data. This simulates different hedging strategies, luck and timing of fuel purchasing.
		\item Outages are modelled by assuming a 93\% outage rate for fuel plants \cite{Ltd2016} and 97\% outage for renewables. \cite{carroll-j}
		\item Generation companies bid their short run marginal costs.
		\item Investments made on highest Net Present Value results. CO2 price, fuel price and demand are predicted 7 years ahead using linear regression. 
		\item Estimated sale of electricity price calculated by simulating a market 7 years into the future with expected power plants that are running and have been taken out of service.
		\item Investors will only invest if they have 25\% of the total upfront costs. (the rest taken on by debt and equity as assumed by WACC value.)
		\item Intermittent power generators can only submit a certain percentage of their total capacity for each load segment. This percentage is matched with empirical data.
		\item Bids accepted by a centralised power exchange based on merit order. Generation companies bid their short run marginal cost.
	\end{itemize}
	\item Assumptions: 
	\begin{itemize}
		\item Yearly time step
		\item Renewables contribute to load curve of each demand segment matched with empirical data of typical wind and solar availability at each demand segment
		\item Different discount rates per user (randomized)
		\item Country initialized with full amount of power plants and generation companies in country and total demand data considered
		\item No curtailment of renewables
		\item Imperfect foresight - Prediction required for demand, co2 price, fuel cost, other investments.
		\item Power plant construction and pre-development periods and costs modelled from UK Government BEIS data
		\item Investments based on highest NPV using a single year 7 (can be changed in scenario file) time steps into the future to predict all years of power plant.
		\item Agents predict next year's fuel, carbon and demand using linear regression and randomized look back period (between 3 and 6.)
		\item Plants are dismantled after their lifetime, and only enter operation after pre-development/construction.
		\item Legacy power plants are reinitialized to random starting year to account for refurbishment.
		\end{itemize}
\end{itemize}