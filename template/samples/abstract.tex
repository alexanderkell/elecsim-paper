Impacts on natural and human systems have already been observed due to anthropogenic greenhouse gas emissions \cite{Masson-Delmotte2018}. To reduce these emissions, a transition to a low-carbon economy is required. Carbon taxes can be used as a tool for pricing in the negative externalities of pollution and enabling a more rapid transition to a low-carbon economy.

This paper proposes the use of agent-based models to simulate an electricity market based in the United Kingdom. We vary carbon tax to observe the effects on investment up until 2050. We find that a carbon tax of \textsterling70 per tonne of \ce{CO2} is sufficient in driving investment to an almost 100\% renewable energy supply. A less aggressive option, however, of setting a carbon tax at \textsterling20 would lead to a 50\% low-carbon, 50\% traditional generation energy mix.

