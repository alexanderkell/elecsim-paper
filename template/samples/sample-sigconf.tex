%
% The first command in your LaTeX source must be the \documentclass command.
\documentclass[sigconf]{acmart}
\usepackage{tabularx}
%
% defining the \BibTeX command - from Oren Patashnik's original BibTeX documentation.
\def\BibTeX{{\rm B\kern-.05em{\sc i\kern-.025em b}\kern-.08emT\kern-.1667em\lower.7ex\hbox{E}\kern-.125emX}}
    
% Rights management information. 
% This information is sent to you when you complete the rights form.
% These commands have SAMPLE values in them; it is your responsibility as an author to replace
% the commands and values with those provided to you when you complete the rights form.
%
% These commands are for a PROCEEDINGS abstract or paper.
\copyrightyear{2018}
\acmYear{2018}
\setcopyright{acmlicensed}
\acmConference[e-Energy '19]{e-Energy '19: The Tenth International Conference on Future Energy Systems}{June 25--28, 2019}{Phoenix, AZ}
\acmPrice{15.00}
\acmDOI{10.1145/1122445.1122456}
\acmISBN{978-1-4503-9999-9/18/06}

%
% These commands are for a JOURNAL article.
%\setcopyright{acmcopyright}
%\acmJournal{TOG}
%\acmYear{2018}\acmVolume{37}\acmNumber{4}\acmArticle{111}\acmMonth{8}
%\acmDOI{10.1145/1122445.1122456}

%
% Submission ID. 
% Use this when submitting an article to a sponsored event. You'll receive a unique submission ID from the organizers
% of the event, and this ID should be used as the parameter to this command.
%\acmSubmissionID{123-A56-BU3}

%
% The majority of ACM publications use numbered citations and references. If you are preparing content for an event
% sponsored by ACM SIGGRAPH, you must use the "author year" style of citations and references. Uncommenting
% the next command will enable that style.
%\citestyle{acmauthoryear}

%
% end of the preamble, start of the body of the document source.
\begin{document}

%
% The "title" command has an optional parameter, allowing the author to define a "short title" to be used in page headers.
\title{ElecSIM: Stochastic Agent-Based Model to Inform Policy for Long-Term Electricity Planning}

%
% The "author" command and its associated commands are used to define the authors and their affiliations.
% Of note is the shared affiliation of the first two authors, and the "authornote" and "authornotemark" commands
% used to denote shared contribution to the research.


\author{Alexander Kell}
\affiliation{%
  \department{School of Computing}
  \institution{Newcastle University}
  \city{Newcastle upon Tyne}
  \country{UK}
}
\email{a.kell2@newcastle.ac.uk}

\author{Matthew Forshaw}
\affiliation{%
  \department{School of Computing}
  \institution{Newcastle University}
  \city{Newcastle upon Tyne}
  \country{UK}
}
\email{matthew.forshaw@newcastle.ac.uk}

\author{A. Stephen McGough}
\affiliation{%
  \department{School of Computing}
  \institution{Newcastle University}
  \city{Newcastle upon Tyne}
  \country{UK}
}
\email{stephen.mcgough@newcastle.ac.uk}
 
%
% By default, the full list of authors will be used in the page headers. Often, this list is too long, and will overlap
% other information printed in the page headers. This command allows the author to define a more concise list
% of authors' names for this purpose.
\renewcommand{\shortauthors}{Kell et al.}

%
% The abstract is a short summary of the work to be presented in the article.
\begin{abstract}

Due to the threat of climate change, a shift from a fossil-fuel based system to one based on renewable energy is required. However, this is not as simple as instantaneously closing down all fossil fuel energy generation and replacing them with renewable sources -- careful decisions need to be taken. To aid decision makers, we present a new tool, ElecSIM, an agent-based modelling framework used to examine the effect of policy on long term investment decisions. We review different techniques currently used to model long term energy decisions, and motivate why agent-based models will become an important strategic tool for policy makers.

We show how modelling stochasticity improves model reliability and motivate why an open-source toolkit is required and demonstrate how ElecSIM meets the requirements of the electricity market. The model runs in yearly time steps, making assumptions based on empirical data on the impact of intermittent renewable energy. We present the dynamics of the system through scenario testing and provide validation. ElecSIM allows non-experts to rapidly prototype new ideas, and is developed around a modular framework -- which allows technical experts to add and remove features at will. 

{\color{red}
This bit seems to be saying "and our work is not complete yet" -- better to say something about the results here (when we have them) and leave the text below for the conclusions.

Future work includes integrating different types of agent based learning for the bidding and investment process, utilising multi-agent reinforcement algorithms that can deal with a non stationary environment. We will use the yearly time-step as a baseline model for integration of a higher temporal and spatial resolution.
}
\end{abstract}

%
% The code below is generated by the tool at http://dl.acm.org/ccs.cfm.
% Please copy and paste the code instead of the example below.
%

%
% Keywords. The author(s) should pick words that accurately describe the work being
% presented. Separate the keywords with commas.

% \keywords{agent-based modelling, simulation, energy market simulation, energy models, policy}

%
% A "teaser" image appears between the author and affiliation information and the body 
% of the document, and typically spans the page. 

%
% This command processes the author and affiliation and title information and builds
% the first part of the formatted document.
\maketitle


\section{Introduction}

The world faces significant challenges from climate change and global warming \cite{Masson-Delmotte2018}. A rise in carbon emissions increases the risk of severe impacts on the world such as rising sea levels, species extinction, heat waves and tropical cyclones \cite{IPCC2014}. The scientific literature concurs that the recent change in climate is anthropogenic, with 97\% of peer reviewed articles of this view \cite{Cook2013}.  

To achieve carbon neutrality, the energy mix must shift from a largely fossil fuel based system, to one based on renewable energy. In essence, using solar, wind and tidal power to generate electricity and power homes, industry and transport \cite{Hoffert2002}. Electricity is a significant proportion of our energy consumption -- consuming 22\% of energy usage per year, which must grow to meet the demands of a low-carbon transport and heating system \cite{Lakshmi2017}. However, although other forms of energy consumption are important we focus here only on the production and consumption of electricity. 


\begin{figure}[b]
	\begin{center}
		\includegraphics[width=0.45\textwidth]{figures/elec_gen_carbon.png}
		\caption{Global electricity generation sources and relative carbon emission intensity. ~\cite{BP2018,Hall1983}}
		\label{fig:fuel_emissions_market_share}
	\end{center}
\end{figure}


For a low carbon energy infrastructure, a transition in the electricity mix is required. Moving from a centralised and homogenous fossil fuel-based system to a distributed system based on renewable energy and batteries. However, such a transition needs to be performed in a safe and non-disruptive manner -- it may be possible to close down all fossil fuel plants in the next year, though if this leads to electricity shortages and power cuts then this is likely to cause significant problems both for companies and homes. Therefore a stepped approach which allows seamless transfer is desirable. This may seem a simple process to achieve -- slowly phase out existing fossil fuel generators and replace by renewable sources -- however, there are many risks and uncertainties in this process. Existing power plants have an expected lifetime and their owners wish to maximise this and the profits which can be made from them, renewable sources are still developing meaning that their efficiency and reliability will change in years to come, along with the fact that most renewable sources are effected by conditions outside the control of the owners (e.g. time of day, wind speed and cloud cover) thus leading to a need for electricity storage. 

To better understand the risks and uncertainties surrounding this transition, and to model the potential actions that can be taken by policy makers, this paper presents ElecSIM, an open source agent-based modelling toolkit, written in Python, which allows for the evaluation of alternative scenarios prior to implementation of policy. Through simulation we can evaluate many strategies in order to identify those most likely to achieve our requirements of rapid but non-disruptive migration from fossil to renewable.

This tool can be used by:
\begin{itemize}
	\item {\bf Policy experts} to test policy outcomes under different scenarios and provide quantitative advice to policy makers. They can provide a simple script defining the policies they wish to use along with the parameters for these polices.
	\item {\bf Energy market developers} who can use the extensible framework to add such things as new energy sources, policy types, consumer profiles and storage types. Thus allowing ElecSIM to adapt to a changing ecosystem.
\end{itemize}
International agreements such as the Paris climate agreement \cite{May2002}, where nation states agreed on the goal of limiting the rise in global average temperature to well below 2$^\circ$C above pre-industrial levels, mean that an open-source, reproducible and transparent model that can be utilised by experts and understood by non-experts is of importance. This allows for the development of policies based on known assumptions, thorough testing and validation.

Mathematical optimisation is often used to determine the least-cost energy infrastructure to attain specified goals \cite{Papadelis2012}. For example, calculating the optimum mix of power plant types to attain the cheapest electricity supply. Optimisation models, therefore, provide information for governments to make investment decisions in power generators over a long-term time scale. 

However, in many Western democracies, the government has liberalised energy markets, with control given to heterogeneous, private investor companies. Agent-based modelling offers a way to model these heterogeneous investor agents, and observe changes in investment decisions based on policies such as carbon tax or subsidies.

Due to the long construction times, long operating periods and high costs of power plants, investment decisions can impact electricity supply over a long time scale \cite{Chappin2017}. Governments, and society, therefore have a role in ensuring that the negative externalities of pollution and carbon emission are priced into electricity generation so that optimal decisions are made. Due to the absence of central control in electricity generation investment, other methods must be used to influence the independent players of the electricity market. Methods such as carbon taxes, policy and regulation can aid in the goals of reducing carbon emissions to limit global warming, as agreed in the Paris agreement \cite{May2002}.

 {\color{red} A diagram showing the different players, who can influence them and how?}

This paper details our model, ElecSIM. Section \ref{Literature Review} is a literature review of the models currently used in practice. Section \ref{Model} details the model and assumptions made, and section \ref{Valdiation and Performance} details how we validated our model, and displays performance metrics. Section \ref{Scenario Testing} details our results, and explores ways in which ElecSIM can be used. We conclude the work and propose future work in section \ref{Conclusion}



%\begin{itemize}
%	\item We have developed a framework for evaluating alternative scenarios, prior to implementation of policy.
%	\item Used by experts working in collaboration with policy makers.
%	\item Importance of a transition in electricity infrastructure (Paris agreement, UK Climate change act)
%	\item Importance of understanding effect of decisions made today on the future (limit of 1.5C by 2050)
%	\item Introduce ElecSIM as a toolkit to inform long-term domestic policy questions in the electricity market. 
%	\item Ability to model the effects of carbon taxation, and the effect of different scenarios 
%	\item Talk about the need to model a non-stationary, dynamic system, with multiple interacting agents with imperfect information
%	\item Requirement for an Open-Source, free Toolkit written in python. Low barrier of entry, and integration with existing python data analytics and machine learning techniques. Transparent, reproducible, and data made available. This allows for results to be open to greater criticism and better inform policy decisions.
%	\item Simple model which matches real life behaviour for low complexity and therefore increases transparency.
%\end{itemize}



\section{Literature Review}\label{Literature Review}
Live experimentation of physical processes is often not practical. The costs of real life experimentation can be prohibitively high, and it normally requires significant time in order to fully ascertain the long-term trends. There is also a risk that changes can have detrimental impacts, and therefore often leads to only minor tweaks being made ~\cite{Forshaw2016}. These factors are particularly true for an electricity market, where decisions made can have long term impacts on energy mix, carbon emissions and investment decisions, with energy plants often having a lifetime of 25 years.  A solution to this is simulation, which can be used for rapid testing and prototyping of ideas. Simulation is the substitution of a physical process with a computer model. The computer model is parametrised by real world data and phenomena. The user is then able to experiment using this model, and assess the likelihoods of outcomes under certain scenarios and input variables \cite{Law:603360}.

\begin{table*}[]
	\begin{tabular}{|l|c|c|c|c|c|}
		\hline
		\multicolumn{1}{|c|}{\textbf{Tool name}} & \textbf{Open Source} & \textbf{Long-Term Investment} & \textbf{Market} & \textbf{Stochastic Inputs} & \textbf{Country Generalisability} \\ \hline
		SEPIA                                    & \checkmark           & x                             & \checkmark      & Demand                     & \checkmark                        \\ \hline
		EMCAS                                    & x                    & \checkmark                    & \checkmark      & Outages                    & \checkmark                        \\ \hline
		NEMSIM                                   & ?              & \checkmark                    & \checkmark      & x                          & x                                 \\ \hline
		AMES                                     & \checkmark           & x                             & Day-ahead       & x                          & x                                 \\ \hline
		PowerACE                                 & x                    & \checkmark                    & \checkmark      & Outages/Demand             & \checkmark                        \\ \hline
		MACSEM                                   & ?              & x                             & \checkmark      & x                          & \checkmark                        \\ \hline
		GAPEX                                    & ?              & x                             & Day-ahead       & x                          & \checkmark                        \\ \hline
		EMLab                                    & \checkmark           & \checkmark                    & Futures         & x                          & \checkmark                        \\ \hline
		ElecSIM                                  & \checkmark           & \checkmark                    & Futures         & \checkmark                 & \checkmark                        \\ \hline
	\end{tabular}
	\caption{Features of electricity market agent based model tools.}
	\label{table:tool_comparison}
\end{table*}


Electricity energy policy modelling is an example where simulation can be used. Real-life experimentation of energy policy is not always feasible due to the long times required to observe results and high risks associated with setting a sub-optimal policy which could radically alter business models and lead to blackouts in electricity supply. Decisions can have long-term impacts, such as producing an electricity market with many expensive and highly polluting coal power plants, that have ramp-up times that are not suitable to accommodate the intermittent electrical flow of renewables. A number of different simulations and computer models have been used to aid policy makers and energy market developers in coming to informed conclusions:

Energy models can typically be classified as top-down macro-economic models or bottom-up techno-economic models~\cite{Bohringer1998}. Top-down models typically focus on behavioural realism with a focus on macro-economic metrics. They are useful for studying economy-wide responses to policies ~\cite{Hall2016}, for example MARKAL-MACRO \cite{Fishbone1981} and LEAP \cite{Heaps2016}. Bottom-up models represent the energy sector in detail, and are written as mathematical programming problems~\cite{Gargiulo2013}. They detail technology explicitly, and can include cost and emissions implications~\cite{Hall2016}.

It is possible to further categorise bottom-up models into optimisation and simulation models. Optimisation energy models minimise costs or maximise welfare from the perspective of a central planner, for instance a government~\cite{Keles2017}. A use-case would be a government that wants cheap, reliable and a low-carbon electricity supply by a specified date. An optimisation model would find the optimal mix of generators to meet this whilst taking into account the constraints such as space, resources and demand. Examples of optimisation models are MARKAL/TIMES~\cite{Fishbone1981} and MESSAGE~\cite{Schrattenholzer1981}. MARKAL is possibly the most widely used general purpose energy systems model~\cite{Pfenninger2014}.

However, electricity market liberalisation in many Western democracies has changed the framework conditions. Centralised, monopolistic, decision making entities have given way to multiple heterogeneous agents acting in their own best interest~\cite{Most2010}. Therefore, certain policy options which encourage changes must be used by a central planner to attain a desired outcome, for example carbon taxes or subsidies. It is proposed that these complex agents are modelled using agent-based simulation, which allows for the modelling of heterogeneous actors.

Agent-based simulation for electricity markets has received increasing attention in recent years, and a number of simulation tools have emerged, for example SEPIA~\cite{Harp2000} EMCAS~\cite{Conzelmann}, NEMSIM~\cite{Batten2006}, AMES~\cite{Sun2007}, PowerACE~\cite{Rothengatter2007}, ~\cite{Praca2003}, GAPEX~\cite{Cincotti2013} and  EMLab~\cite{Chappin2017}. There are numerous different electricity markets that models can focus on, for example, pool markets, bilateral trading, day-ahead and futures markets.

However, by referring to Table \ref{table:tool_comparison}, it can be seen that none of these suit the needs of an open source, long-term market model. The inclusion of stochastic input variables allows for better performance.

SEPIA \cite{Harp2000} is a discrete event agent based model which utilises Q-learning to model the bids made by GenCos. SEPIA models plants as being always on, and does not have an independent system operator (ISO), which in an electricity market, is an independent non-profit organization for coordinating and controlling of regular operations of the electric power system and market  \cite{Zhou2007}. SEPIA does not model a spot market, instead focusing on bilateral contracts. As opposed to this, ElecSIM has been designed with a merit-order, spot market in mind where renewable energy runs intermittently.

MACSEM \cite{Praca2003} simulates a bilateral and pool market. It has been used to probe the effects of market rules and conditions by simulating and testing different bidding strategies. However, MACSEM does not model long term investment decisions.

EMCAS ~\cite{Conzelmann} is a closed source agent-based framework which investigates the interactions between physical infrastructures and economic behaviour of market participants. ElecSIM, however, focuses on purely the dynamics on the market, with an aim of providing a simplified, transparent, open source model of market operation, whilst maintaining robustness.

PowerACE ~\cite{Rothengatter2007} is also a closed source agent-based simulation of electricity markets that integrates short-term perspectives of daily electricity trading and long-term investment decisions. Similarly to ElecSIM, PowerACE initialises agents with all power plants in their respective country. However, unlike ElecSIM, PowerACE does not take into account stochasticity of price risks in electricity markets which is of crucial importance to real markets~\cite{Most2010}.

EMLab ~\cite{Chappin2017} is also an agent-based modelling toolkit for the electricity market. EMLab models an endogenous European emissions trading scheme with a yearly time-step. However, like PowerACE, EMLab differs from ElecSIM by not taking into account stochasticity in the electricity markets, such as outages, differing fuel prices within a year period and stochasticity in power plant operating costs. However, after correspondence with the authors, we were unable to run EMLab.

AMES ~\cite{Sun2007} is an agent-based model specific to the US Wholesale Power Market Platform. GAPEX \cite{Cincotti2013} is an agent-based framework for modelling and simulating power exchanges in MATLAB . GAPEX utilises an enhanced version of the reinforcement technique Roth-Erev to consider the presence of affine total cost functions. However, neither of these model the long-term dynamics that ElecSIM is designed for.

Table \ref{table:tool_comparison} shows the features of each of the tools reviewed in this section. We propose ElecSIM to fill the gaps that are not currently covered, which includes an open source long-term stochastic investment model. 


%\begin{itemize}
%	\item Agent Based Models - eg. EMCAS, PowerACE, EMLab: Leaves a requirement for an open source toolkit written in python. Many one-off models available, however difficult to apply to different scenarios.
%	(SEPIA [6], EMCAS [7], NEMSIM [8], AMES [9], PowerACE [10], MASCEM [11, 12], and GAPEX [13] \cite{Lopes})
%	\item Bottom-up optimization models to find minimum cost of electricity system. \cite{Pfenninger2014}. eg. MARKAL/TIMES, MESSAGE. (These do not provide information on how to achieve a certain goal, particularly in a liberalized energy market. Or scenarios as to why a goal may not be achieved as the goal is assumed to be achieved.)
%	\item Computational general equilibrium (CGE) models - Top-down macroeconomic models partial equilibrium model (energy supply, demand, cross-border trade, emissions)- Can be highly complex and difficult to understand. eg. NEMS, PRIMES.
%\end{itemize}




\section{ElecSIM Architecture} \label{Model}

\begin{figure}
	\centering
	\includegraphics[width=0.97\linewidth]{figures/System_overview_large}
	\caption{System overview of agent-based market model.}
	\label{fig:systemoverview}
\vskip -6mm
\end{figure}


\begin{figure*}[h]
	\centering
	\begin{subfigure}[b]{0.33\textwidth}
		\centering
		\includegraphics[width=\textwidth]{figures/scenarios/demand099-carbon10-datetime.png}
		\caption[Network2]%
		{\small \textsterling10 carbon tax.}
		\label{fig:demand99carbon10}
	\end{subfigure}
	\hfill
	\begin{subfigure}[b]{0.33\textwidth}  
		\centering 
		\includegraphics[width=\textwidth]{figures/scenarios/demand099-carbon20-datetime.png}
		\caption[]%
		{\textsterling20 carbon tax.}
		\label{fig:demand99carbon20}
	\end{subfigure}
	\begin{subfigure}[b]{0.33\textwidth}
		\centering
		\includegraphics[width=\textwidth]{figures/scenarios/demand099-carbon70-datetime.png}
		\caption[Network2]%
		{\small \textsterling70 carbon tax.}
		\label{fig:demand99carbon70}
	\end{subfigure}
	\caption{Scenarios from 2020 to 2050 with varying carbon tax.}
\end{figure*}


The agent-based model is made up of five significant parts: the agents which are made up of the generation companies (GenCos) and demand agents; power plants and a market operator which controls the spot market. How these parts interact are displayed in Figure \ref{fig:systemoverview}. The relevant data sources are also provided here.

We initialise the United Kingdom with our model with exemplar data from the UK. We model every single power plant in operation in the year 2018, which are owned by their respective generation companies. Individual historical power plant costs are estimated from levelized cost of electricity (LCOE) \cite{Dale2013, IEA2015,IRENA2018}, whereas future and present power plant costs are taken from the department of business and industrial strategy \cite{Department2016}. The variable operation and maintenance cost was defined stochastically to model the changing costs per project. A uniform distribution was chosen to provide sufficient variance between projects.

The demand agent is modelled as a single aggregated demand, split up into 20 segments of a load duration curve (LDC), enabling us to increase speed of computation whilst maintaining accuracy. An LDC is defined as a yearly load sorted in order of magnitude. 

We model the influence of outages using availability data for gas, coal, photovoltaic, offshore and onshore power generators \cite{Ltd2016, Hunt2015, carroll-j}. Historical availabilities are modelled for older gas, coal and hydro power plants \cite{AlbertaSystemElectricOperator2016}. Capacity factors were taken as an average of the UK for solar and wind \cite{Pfenninger2016, Staffell2016}. Where capacity factors is defined as the ratio of electrical output over a given time period over the maximum possible electrical energy output. 

The generation companies make electricity bids each year for each of their power plants. The market operator then matches demand with supply in order of price, also known as merit-order dispatch. We model a uniform pricing market, where each of the companies are paid the highest accepted bid.

GenCos have the ability to invest every year in new power plants based on the expected net present value (NPV) of each type of power plant. NPV is a summation of the present value of a series of present and future cash flow. The NPV calculation is dependent on a stochastic representation of GenCos predictions of fuel, carbon and electricity price and demand.

Each GenCo has a separate weighted average cost of capital (WACC), which is the rate that a company is expected to pay on average for its stock and debt, this is used as the discount rate in the NPV calculation \cite{KincheloeStephenC1990TWAC}. The WACC is modelled as a stochastic variable, with a Gaussian distribution and a $\pm3\%$ standard deviation, with values of 5.9\% for non-nuclear power plants, and 10\% for nuclear power plants \cite{KPMG2017, Paper2012}. 

The model took yearly time-steps to limit the impact on computation time, however, to model the intermittency of renewable generation, we correlated demand with the respective capacity factor, enabling for example, solar and wind to only contribute a certain capacity to their load curve.

Stochasticity of fuel price within a year was also modelled, to take into account difference in hedging strategies and chance. An ARIMA model \cite{ARIMA} was fit to historic coal and natural gas prices.





\section{Validation and Performance}\label{Valdiation and Performance}

Validation of models is important to ascertain that the results are accurate. However, it should be noted that these long-term simulations are not predictions of the future, rather possible outcomes based upon certain assumptions. Therefore, the results from ElecSIM should be analysed by understanding the underlying assumptions of the model, and comparing inputs to outcomes.

Jager posits that a certain outcome or development path, captured by empirical data, might have developed in a completely different direction due to chance \cite{Jager2006a}. However, through observation, the processes that emerge from a model should be realistic and in keeping with expected behaviour \cite{Jager2006}.

We begin by comparing the price duration curve in the year 2018. Figure \ref{fig:price_duration_curve} shows the N2EX Day Ahead Auction Prices of the UK \cite{nordpool_2019}, the stochastic simulated electricity prices, and the non-stochastic electricity price throughout the year 2018.

Table \ref{table:validation_metrics} shows performance metrics of the stochastic and non-stochastic runs versus the actual price duration curve. It can be seen that stochastic implementation (ElecSIM), improves the mean absolute error (MAE) by $52.5\%$.

Therefore, the adding of stochasticity to fuel prices and variable operation \& maintenance improves on previous attempts of a yearly step model.

By observing the processes that emerge from the long-term scenarios, we can see that carbon price and investment in renewable generation are positively correlated, and is what one would expect.

We found that the net present value (NPV) calculations are realistic, with onshore wind and Combined Cycle Gas Turbines (CCGT) the technologies that are most invested in. It is true, within the United Kingdom, that Onshore wind and CCGT power generators are the most cost effective, and heavy government subsidies are required for other generation types such as nuclear and coal. 

\begin{figure}[H]
	\begin{center}
		\includegraphics[width=0.5\textwidth]{figures/load_price_duration_curve_comparison.pdf}
		\caption{Price duration curve which compares real electricity prices to those paid in ElecSIM with stochasticity (40 runs) and no stochasticity.}
		\label{fig:price_duration_curve}
	\end{center}
\end{figure}

\begin{table}[h]
	\centering
	\csvautobooktabular{tables/validation/initialisation_run_validation.csv}
	\caption{Validation performance metrics.}
	\label{table:validation_metrics}
\end{table}



\begin{itemize}
	\item Validation of model 
	\begin{itemize}
		\item Compare price duration curve
		\item Compare power plant costs and NPV calculations
		\item Look number of steps ahead to compare electricity mix and compare to actual (cross-validation)
	\end{itemize} 
	\item Performance metrics - Comparison with EMLab, PowerACE (15 minute run time)
	\begin{itemize}
		\item Memory, disk size, runtime
		\item Increase in time complexity with additional data.
	\end{itemize}
\end{itemize}


\section{Scenario Testing}\label{Scenario Testing}

This section describes scenario runs using ElecSIM. Here, we vary the carbon tax and either grow or reduce total electricity demand. This was done to observe the effects of carbon tax policy on long-term investment.

ElecSIM was built using python, this enabled us to lower barriers to entry and allow for users to integrate state-of-the-art machine learning and statistical packages in future work. We used project mesa as an open source agent based modelling framework for its ease of use \cite{Masad2015}.

{\color{blue}
We assume that carbon tax is set by the government, and not subject to market forces such as the EU Emissions Trading Scheme \cite{Council2016}.

We run 16 different scenarios 8 times each, with demand increasing and decreasing by 1\% per year and  varying carbon prices. In this section we explore a decreasing demand of 1\% a year. We chose this due to the increasing efficiency of homes, industry and technology, and due to the recent trend in the UK. Demand, however, did not display a large effect on the optimum carbon price. We select a burn-in period of 6 years, due to the fact that the majority of power plants take 6 years to go from investment to operation.

Table \ref{table:scenario_statistics}, in the appendix, displays the summary statistics of each run.

It can be seen from Figure \ref{fig:demand99carbon10} that a carbon tax of \textsterling10 per year does little to influence investment in low-carbon, renewable technology. With traditional, fossil fuel based generation, providing the majority of supply in each year. However, there is an increase in renewable technology over the years, starting from mean 15.85\% market share in the year range 2019-2029, to 24.38\% in the year range 2039-2050. A similar increase of renewable energy with a carbon tax of \textsterling0 can be seen, albeit at a lower mean by the year range 2039-2050 (22.29\%).
}

The UK Government BEIS have predicted a carbon tax increasing from \textsterling18 to \textsterling200 by 2050. With carbon price increasingly linearly from 2030 to 2050. We have approximated these assumptions in Figure \ref{fig:demand99carbon18} and modelled the results. Interestingly, the results show only a slight increase in low-carbon supply over the \textsterling20 carbon tax energy mix. This demonstrates the importance of long-term modelling, and understanding the long-term impacts that can result due to today's decisions.

It is hypothesised that a lower carbon tax early on changes the market dynamics for years to come, due to certain price structures, and therefore it takes a long time for renewable energy to recover.

Figure \ref{fig:demand99carbon40} shows that a carbon tax of \textsterling40 is sufficient in beginning to move towards a low-carbon economy, with backup fossil fuel generators.

However, by referring to Figure \ref{fig:demand99carbon70} it can be seen that to have 100\% renewable, a carbon price of \textsterling70 is required. 

These results show the importance of making difficult decisions as soon as possible to have the biggest effect on the energy mix for years to come.

\begin{figure}
	\begin{center}
		\includegraphics[width=0.5\textwidth]{figures/scenarios/demand099-carbon70-datetime.png}
		\caption{{\color{blue}Demand decreasing by 1\% per year with a carbon tax of \textsterling70.}}
		\label{fig:demand99carbon70}
	\end{center}
\end{figure}



\begin{figure*}[h]
	\centering
	\begin{subfigure}[b]{0.475\textwidth}
		\centering
		\includegraphics[width=\textwidth]{figures/scenarios/demand099-carbon10-datetime.png}
		\caption[Network2]%
		{{{\color{blue}{\small \textsterling10 carbon tax.}}}}
		\label{fig:demand99carbon10}
	\end{subfigure}
	\hfill
	\begin{subfigure}[b]{0.475\textwidth}  
		\centering 
		\includegraphics[width=\textwidth]{figures/scenarios/demand099-carbon20-datetime.png}
		\caption[]%
		{{{\color{blue}\textsterling20 carbon tax.}}}
		\label{fig:demand99carbon20}
	\end{subfigure}
	\vskip\baselineskip

	\quad
	
\end{figure*}
%\FloatBarrier








%\begin{figure}[h]
%	\begin{center}
%		\includegraphics[width=0.5\textwidth]{figures/scenarios/demand099-carbon10-datetime.png}
%		\caption{Demand decreasing by 1\% per year and a carbon tax of \textsterling10}
%		\label{fig:demand99carbon10}
%	\end{center}
%\end{figure}
%
%\begin{figure}[h]
%	\begin{center}
%		\includegraphics[width=0.5\textwidth]{figures/scenarios/demand099-carbon20-datetime.png}
%		\caption{Demand decreasing by 1\% per year and a carbon tax of \textsterling20}
%		\label{fig:demand99carbon10}
%	\end{center}
%\end{figure}
%
%
%
%\begin{figure}[h]
%	\begin{center}
%		\includegraphics[width=0.5\textwidth]{figures/scenarios/demand099-carbon18-datetime.png}
%		\caption{Demand decreasing by 1\% per year and a carbon tax of \textsterling20}
%		\label{fig:demand99carbon10}
%	\end{center}
%\end{figure}
%
%
%
%\begin{figure}[h]
%	\begin{center}
%		\includegraphics[width=0.5\textwidth]{figures/scenarios/demand099-carbon40-datetime.png}
%		\caption{Demand decreasing by 1\% per year and a carbon tax of \textsterling20}
%		\label{fig:demand99carbon10}
%	\end{center}
%\end{figure}











%\begin{itemize}
%	\item Effect of different carbon tax on investments made.
%	\item Effects of different demand scenarios. (High peaks, high growth, high reduction in demand)
%	\item Effects of high fuel prices.
%	\item Different costs of capital (eg. Borrowing for Nuclear of interest rate to equal 2\% at government bonds rate, as opposed to 10\% for private companies.)
%	\item Different learning rates for renewable costs.
%	\item The effect of long term carbon tax policy (eg. Carbon price known for next 25 years) vs short term changes in carbon tax.
%\end{itemize}




\section{Conclusions}\label{Conclusion}

Agent-based models provide a method of simulating investor behaviour in an electricity market. We observed that an increase in carbon tax had a significant impact on investment. These findings enable policy makers to better understand the impact that their decisions may have. For a high uptake of renewable energy technology, rapid results can be seen after 10 years with a carbon tax of \textsterling70 (\$90).

Agent-based models open up the possibility of testing differing investor behaviours through techniques such as reinforcement learning. This can be extended to incorporate collusion which can have an impact in liberalized electricity markets \cite{Benjamin2016}.

%We propose the integration of a higher temporal and spatial resolution to model the utility of batteries, distributed generation and scarceness in renewable resources such as wind and solar at certain times of the day.


\FloatBarrier




%
% The acknowledgments section is defined using the "acks" environment (and NOT an unnumbered section). This ensures
% the proper identification of the section in the article metadata, and the consistent spelling of the heading.
\begin{acks}
To Robert, for the bagels and explaining CMYK and color spaces.
\end{acks}

%
% The next two lines define the bibliography style to be used, and the bibliography file.
\bibliographystyle{ACM-Reference-Format}
\bibliography{library,custombibtex}

% 
% If your work has an appendix, this is the place to put it.
\appendix

\section{Research Methods}

\subsection{Parameters}

\begin{table*}[]
	\begin{tabularx}{\linewidth}{|p{\dimexpr.5\linewidth-40\tabcolsep-1.3333\arrayrulewidth}|l|l|l|l|l|l|l|l|l|l|l|l|l|}
\hline
Type                          & Plant\_Size & year             & $\eta$ & $OP$ & $P_D$ & $C_D$ & $P_C$    & $C_C$     & $I_C$    & $F_C$   & $V_C$ & $In_C$  & $Con_C$  \\ \hline
CCGT                          & 168.0       & 2018, 2020, 2025 & 0.34   & 25.0 & 3     & 3     & 60000.0  & 700000.0  & 13600.0  & 28200.0 & 5.0   & 2900.0  & 3300.0   \\ \hline
CCGT                          & 1200.0      & 2018, 2020, 2025 & 0.54   & 25.0 & 3     & 3     & 10000.0  & 500000.0  & 15100.0  & 12200.0 & 3.0   & 2100.0  & 3300.0   \\ \hline
CCGT                          & 1471.0      & 2018, 2020, 2025 & 0.53   & 25.0 & 3     & 3     & 10000.0  & 500000.0  & 15100.0  & 11400.0 & 3.0   & 1900.0  & 3300.0   \\ \hline
Coal                          & 552.0       & 2025             & 0.32   & 25.0 & 6     & 6     & 40000.0  & 3400000.0 & 10000.0  & 68200.0 & 6.0   & 13000.0 & 3800.0   \\ \hline
Coal                          & 624.0       & 2025             & 0.32   & 25.0 & 5     & 5     & 70000.0  & 4200000.0 & 10000.0  & 79600.0 & 3.0   & 19300.0 & 3800.0   \\ \hline
Coal                          & 652.0       & 2025             & 0.3    & 25.0 & 5     & 5     & 60000.0  & 3900000.0 & 10000.0  & 65300.0 & 5.0   & 22700.0 & 3800.0   \\ \hline
Coal                          & 734.0       & 2025             & 0.38   & 25.0 & 5     & 5     & 60000.0  & 2600000.0 & 10000.0  & 56400.0 & 3.0   & 9600.0  & 3800.0   \\ \hline
Coal                          & 760.0       & 2025             & 0.35   & 25.0 & 5     & 5     & 40000.0  & 2800000.0 & 10000.0  & 52100.0 & 5.0   & 14000.0 & 3800.0   \\ \hline
Hydro                         & 0.033       & 2018, 2020, 2025 & 1.0    & 35.0 & 0     & 0     & 0.0      & 6300000.0 & 0.0      & 83300.0 & 0.0   & 0.0     & 0.0      \\ \hline
Hydro                         & 1.046       & 2018, 2020, 2025 & 1.0    & 35.0 & 0     & 0     & 0.0      & 3300000.0 & 400.0    & 18200.0 & 0.0   & 0.0     & 0.0      \\ \hline
Hydro                         & 11.0        & 2018, 2020, 2025 & 1.0    & 41.0 & 2     & 2     & 60000.0  & 3000000.0 & 0.0      & 45100.0 & 6.0   & 0.0     & 0.0      \\ \hline
Nuclear                       & 3300.0      & 2025             & 1.0    & 60.0 & 5     & 8     & 240000.0 & 4100000.0 & 11500.0  & 72900.0 & 5.0   & 10000.0 & 500.0    \\ \hline
OCGT                          & 96.0        & 2018, 2020, 2025 & 0.35   & 25.0 & 2     & 2     & 80000.0  & 600000.0  & 12600.0  & 9900.0  & 4.0   & 2500.0  & 2400.0   \\ \hline
OCGT                          & 299.0       & 2018, 2020, 2025 & 0.35   & 25.0 & 2     & 2     & 30000.0  & 400000.0  & 13600.0  & 9600.0  & 3.0   & 1600.0  & 2500.0   \\ \hline
OCGT                          & 311.0       & 2018, 2020, 2025 & 0.35   & 25.0 & 2     & 2     & 30000.0  & 400000.0  & 13600.0  & 9500.0  & 3.0   & 1600.0  & 2500.0   \\ \hline
OCGT                          & 400.0       & 2018, 2020, 2025 & 0.34   & 25.0 & 2     & 2     & 30000.0  & 300000.0  & 15100.0  & 7800.0  & 3.0   & 1300.0  & 2500.0   \\ \hline
OCGT                          & 625.0       & 2018, 2020, 2025 & 0.35   & 25.0 & 2     & 2     & 20000.0  & 300000.0  & 15100.0  & 4600.0  & 3.0   & 1200.0  & 2400.0   \\ \hline
OCGT\_CCS                     & 290.0       & 2025             & 0.24   & 25.0 & 5     & 5     & 80000.0  & 2300000.0 & 15100.0  & 31800.0 & 3.0   & 8500.0  & 2500.0   \\ \hline
Offshore                      & 321.0       & 2018             & 0.0    & 23.0 & 5     & 3     & 60000.0  & 2200000.0 & 69300.0  & 30900.0 & 3.0   & 1400.0  & 33500.0  \\ \hline
Offshore                      & 321.0       & 2020             & 0.0    & 23.0 & 5     & 3     & 60000.0  & 2100000.0 & 69300.0  & 30000.0 & 3.0   & 1400.0  & 32600.0  \\ \hline
Offshore                      & 321.0       & 2025             & 0.0    & 23.0 & 5     & 3     & 60000.0  & 1900000.0 & 69300.0  & 28600.0 & 3.0   & 1300.0  & 31100.0  \\ \hline
Offshore                      & 844.0       & 2018             & 0.0    & 22.0 & 5     & 3     & 120000.0 & 2400000.0 & 323000.0 & 48600.0 & 4.0   & 3300.0  & 50300.0  \\ \hline
Offshore                      & 844.0       & 2020             & 0.0    & 22.0 & 5     & 3     & 120000.0 & 2300000.0 & 323000.0 & 47300.0 & 3.0   & 3300.0  & 48900.0  \\ \hline
Offshore                      & 844.0       & 2025             & 0.0    & 22.0 & 5     & 3     & 120000.0 & 2100000.0 & 323000.0 & 45400.0 & 3.0   & 3100.0  & 47000.0  \\ \hline
Onshore                       & 0.01        & 2018             & 1.0    & 20.0 & 0     & 0     & 0.0      & 3700000.0 & 0.0      & 29700.0 & 0.0   & 0.0     & 0.0      \\ \hline
Onshore                       & 0.01        & 2020             & 1.0    & 20.0 & 0     & 0     & 0.0      & 3600000.0 & 0.0      & 29600.0 & 0.0   & 0.0     & 0.0      \\ \hline
Onshore                       & 0.01        & 2025             & 1.0    & 20.0 & 0     & 0     & 0.0      & 3500000.0 & 0.0      & 29600.0 & 0.0   & 0.0     & 0.0      \\ \hline
Onshore                       & 0.482       & 2018             & 1.0    & 20.0 & 0     & 0     & 0.0      & 2200000.0 & 200.0    & 56900.0 & 0.0   & 0.0     & 0.0      \\ \hline
Onshore                       & 0.482       & 2020             & 1.0    & 20.0 & 0     & 0     & 0.0      & 2100000.0 & 200.0    & 56900.0 & 0.0   & 0.0     & 0.0      \\ \hline
Onshore                       & 0.482       & 2025             & 1.0    & 20.0 & 0     & 0     & 0.0      & 2000000.0 & 200.0    & 56700.0 & 0.0   & 0.0     & 0.0      \\ \hline
Onshore                       & 20.0        & 2018             & 0.0    & 24.0 & 4     & 2     & 110000.0 & 1200000.0 & 3300.0   & 23200.0 & 5.0   & 1400.0  & 3100.0   \\ \hline
Onshore                       & 20.0        & 2020             & 0.0    & 24.0 & 4     & 2     & 110000.0 & 1200000.0 & 3300.0   & 23000.0 & 5.0   & 1400.0  & 3100.0   \\ \hline
Onshore                       & 20.0        & 2025             & 0.0    & 24.0 & 4     & 2     & 110000.0 & 1200000.0 & 3300.0   & 22400.0 & 5.0   & 1400.0  & 3000.0   \\ \hline
PV                            & 0.003       & 2018             & 1.0    & 30.0 & 0     & 0     & 0.0      & 1500000.0 & 0.0      & 23500.0 & 0.0   & 0.0     & 0.0      \\ \hline
PV                            & 0.003       & 2020             & 1.0    & 30.0 & 0     & 0     & 0.0      & 1500000.0 & 0.0      & 23400.0 & 0.0   & 0.0     & 0.0      \\ \hline
PV                            & 0.003       & 2025             & 1.0    & 30.0 & 0     & 0     & 0.0      & 1400000.0 & 0.0      & 23200.0 & 0.0   & 0.0     & 0.0      \\ \hline
PV                            & 0.455       & 2018             & 1.0    & 30.0 & 0     & 0     & 0.0      & 1000000.0 & 200.0    & 9400.0  & 0.0   & 0.0     & 0.0      \\ \hline
PV                            & 0.455       & 2025             & 1.0    & 30.0 & 0     & 0     & 0.0      & 900000.0  & 200.0    & 9200.0  & 0.0   & 0.0     & 0.0      \\ \hline
PV                            & 1.0         & 2018             & 0.0    & 25.0 & 1     & 0     & 20000.0  & 700000.0  & 0.0      & 6600.0  & 3.0   & 2600.0  & 1300.0   \\ \hline
PV                            & 1.0         & 2020             & 0.0    & 25.0 & 1     & 0     & 20000.0  & 700000.0  & 0.0      & 6300.0  & 3.0   & 2600.0  & 1300.0   \\ \hline
PV                            & 1.0         & 2025             & 0.0    & 25.0 & 1     & 0     & 20000.0  & 600000.0  & 0.0      & 5900.0  & 3.0   & 2400.0  & 1200.0   \\ \hline
PV                            & 4.0         & 2018             & 0.0    & 25.0 & 1     & 0     & 60000.0  & 700000.0  & 200.0    & 8300.0  & 0.0   & 1200.0  & 1300.0   \\ \hline
PV                            & 4.0         & 2020             & 0.0    & 25.0 & 1     & 0     & 60000.0  & 700000.0  & 200.0    & 8000.0  & 0.0   & 1100.0  & 1300.0   \\ \hline
PV                            & 4.0         & 2025             & 0.0    & 25.0 & 1     & 0     & 60000.0  & 600000.0  & 200.0    & 7500.0  & 0.0   & 1100.0  & 1200.0   \\ \hline
PV                            & 16.0        & 2018             & 0.0    & 25.0 & 1     & 0     & 70000.0  & 700000.0  & 400.0    & 5600.0  & 0.0   & 2000.0  & 1300.0   \\ \hline
PV                            & 16.0        & 2020             & 0.0    & 25.0 & 1     & 0     & 70000.0  & 600000.0  & 400.0    & 5400.0  & 0.0   & 1900.0  & 1300.0   \\ \hline
PV                            & 16.0        & 2025             & 0.0    & 25.0 & 1     & 0     & 70000.0  & 600000.0  & 400.0    & 5100.0  & 0.0   & 1800.0  & 1200.0   \\ \hline
Reciprocating Engine (Diesel) & 20.0        & 2018, 2020, 2025 & 0.34   & 15.0 & 2     & 1     & 10000.0  & 300000.0  & 2200.0   & 10000.0 & 2.0   & 1000.0  & -31900.0 \\ \hline
Reciprocating Engine (Gas)    & 20.0        & 2018, 2020, 2025 & 0.32   & 15.0 & 2     & 1     & 10000.0  & 300000.0  & 3400.0   & 10000.0 & 2.0   & 1000.0  & -31900.0 \\ \hline
	\end{tabularx}
\end{table*}



\end{document}
