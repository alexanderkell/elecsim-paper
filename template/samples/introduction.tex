
Governmental policy is a tool that can be used to aid in the transition to a low-carbon economy to prevent the worst effects of climate change. Options include a tax on all carbon emissions or subsidies in low-carbon technologies. In this paper, we vary carbon taxes to assess the long-term impacts on investment in the electricity market. We used a general agent-based model simulation made for wholesale electricity markets, created by us. 


Simulation is a technique to create a physical system in a virtual world.  In this context a model is defined as a set of mathematical formulas and algorithms which are designed to mimic real life \cite{Forshaw2016}. Simulation allows practitioners to rapidly prototype high risk ideas in this virtual model and assess their outcome before implementation in the real world.

The electricity market in many western democracies consists of multiple heterogenous actors acting for their own best interest \cite{Most2010}. Agent-based modelling is a technique which allows for the simulation of these heterogenous actors with different risk profiles, profit requirements and preferences. A number of agent-based models have been used to model the impact of carbon tax on long term investments \cite{Tang2015, Chen2014, Chappin2017}. ABMs have been utilised in this field to address phenomena such as market power \cite{Ringler2016a}.

We model the realisation of the wholesale electricity market in the United Kingdom and adjust carbon tax in our agent-based model to see the effect of long-term investment. Whilst we have modelled the United Kingdom, it would be possible to model for any country with different parameters. We posit that decisions made today can have complex long-term consequences, the process of which can be observed through simulation.



This paper details our model and different carbon scenarios. Section \ref{Model} details the model, assumptions made and parameters. Section \ref{Scenario Testing} presents the our results. We conclude our work in Section \ref{Conclusion} and explore possible routes forward .


