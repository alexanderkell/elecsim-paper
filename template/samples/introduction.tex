
Governmental policy is a tool that can be used to aid the transition to a low-carbon economy to prevent the worst effects of climate change. Options include a tax on all carbon emissions or subsidies in low-carbon technologies. In this paper, we propose the varying of carbon taxes to assess the long-term impacts on investment in the electricity market using an agent-based model simulation. 


Simulation is a technique to create a physical system in a virtual model.  In this context a model is defined as a set of mathematical formulas and algorithms which are designed to mimic real life \cite{Forshaw2016}. Simulation allows practitioners to rapidly prototype high risk ideas in this virtual model and assess their outcome before implementation in the real world.

The electricity market in many western democracies consists of multiple heterogenous actors acting for their own best interest \cite{Most2010}. Agent-based modelling is a technique which allows for the simulation of these heterogenous actors with different risk profiles, profit requirements and preferences. A number of agent-based models have been used to model the impact of carbon tax on long term investments \cite{Tang2015, Chen2014, Chappin2017}. ABMs have been utilised in this field to address phenomena such as market power \cite{Ringler2016a}.

We model the realisation of the wholesale electricity market in the United Kingdom and adjust carbon tax in our agent-based model to see the effect of long-term investment. We posit that decisions made today can have complex long-term consequences, the process of which can be observed through simulation.



This paper details our model, ElecSIM. We contribute a new open-source framework, and test different scenarios with varying carbon taxes to provide advice to stakeholders. Section \ref{Literature Review} is a literature review of the models currently used in practice. Section \ref{Model} details the model and assumptions made, and Section \ref{Valdiation and Performance} details how we validated our model, and displays performance metrics. Section \ref{Scenario Testing} details our results, and explores ways in which ElecSIM can be used. We conclude the work and propose future work in Section \ref{Conclusion}.


